% !TeX root = ../main.tex
% Add the above to each chapter to make compiling the PDF easier in some editors.

\chapter{Introduction}\label{chapter:introduction}

It is quite difficult or impossible to find reliable sources for the following story, this is why it can be stamped as an urban legend. The tale is always told a bit different, varying in the acting person's gender, age or the reason they wanted to leaf the driver's seat. Nevertheless, the essence in all versions is the driver steering a new bought van or caravan, having an additional cruise control extra, to hold the speed while driving. For a certain reason the driver wants to pick something up, from the back of the vehicle, so he or she remembers, that there is this new feature installed in their car. The mistake was mixing up cruise control with a fully adequate autopilot, so as they turn on the "auto pilot" and leave their front seat, while driving, of course the vehicle leaves the road and crashes. \newline
This story, was presumably first recorded on paper, by Jan Harold Brunvand in 1984 \cite{brunvand}, however it is still present today and can be found in all varieties on the Internet. While the person's idea of cruise control might have been considered extremely dumb or unworldly, when the story first appeared, this expectation became successively imaginable as the years went by. \newline
In 2004 the first DARPA Grand Challenge took place \cite{darpa2004} in the Mojave desert(USA), proclaimed by the Defense Advanced Research Projects Agency. It was a competition between different teams, competing to build an autonomous driving vehicle, that should be able to drive a predefined course, with a length of 240 kilometers, without any control advices from outside. In total 15 cars competed, due to technical problems none of the teams was capable of completing the full course, in fact the furthest distance driven was a little lesa than 12 kilometers, fortunately this was not the end of these competitions. \newline
The next Grand Challenge was held in 2005, this time 23 cars participated in the race, 22 of them were actually capable of passing the record, set previous year. In fact five vehicles were able to finish the race completely \cite{darpa2005}.\newline
After the success in 2005, the competition evolved into the 2007 DARPA Grand Challenge, also called the DARPA Urban Challenge, because in contrast to the other competitions, the course changed from a track in the desert to a slightly more complicated route, containing actual roads, other cars and traffic regulations. After qualifying phases, eleven teams were allowed to participate in the final event, six of them actually driving through the finishing line \cite{darpa2007}.\newline
Even though these events started over a decade ago, 
these contests can be seen exemplary for the rapidly growing improvements and advancements in developing autonomous driving vehicles. Today, big car companies like BMW, Audi and Tesla are in testing of completely autonomous vehicles, but also tech giants like Google with their Waymo project  started testing these vehicles in 2017 on public roads, without anyone in the driver's seat \footnote{https://waymo.com}. It could be said, with these vehicles, it's now absolutely safe to leave the front seat, to pick something up, from the back. 
