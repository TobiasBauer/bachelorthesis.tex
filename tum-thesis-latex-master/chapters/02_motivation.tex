% !TeX root = ../main.tex
% Add the above to each chapter to make compiling the PDF easier in some editors.

\chapter{Motivation}\label{chapter:motivation}
//Noch in Bearbeitung
In present times advanced driver assistance systems, also called ADAS, are common features, even for middle classed driving vehicles. Fully autonomous cars are on the rise and probably only a matter of years before entering series production and running on public streets. These systems usually cover a certain kind of traffic sign recognition. Right now older or not fully equipped cars often lack these kind of extras, which induces a certain absence of relative safeness over newer cars. As it is not really economic to retrofit these features professionally, the question arises, whether you could add some of them to your vehicle without too much of circumstances, for example with a low cost device, most drivers already have, like an Android Smartphone mounted to the mobile phone bracket. In particular, because these devices have all the hardware, you need to implement this kind of features on, a camera, processors and a display to output the results. So it would actually make sense to have an application, that assists the driver in following the speed limit, when steering a vehicle. \newline
Being open source is one of the major benefits of Android over other operating systems for Handheld devices, meaning the code is publicly visible and primarily it is free of charges. There are also other gratuitous components beneficial to implement such functionalities, like the OpenCV-library for computer vision and other open-source frameworks for machine learning, like Tensorflow or Caffe(2). So the question comes to mind, how well this low-cost kind of solution actually scores. 

\section{KIA4SM}
Cooperative Intelligent Transport Systems (C-ITS) is a current research field, aiming to connect vehicles, mobile devices, traffic and transportation infrastructure\cite{kia4sm}. The KIA4SM project wants to achieve cooperative behavior between independently designed and running systems participating in traffic. Therefore it is not inevitable to change each participating system to make communication possible, rather it is desired to generate a homogeneous platform for heterogeneous
devices
%See~\autoref{tab:sample}, \autoref{fig:sample-drawing}, \autoref{fig:sample-plot}, \autoref{fig:sample-listing}.


