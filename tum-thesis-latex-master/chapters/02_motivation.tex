% !TeX root = ../main.tex
% Add the above to each chapter to make compiling the PDF easier in some editors.

\chapter{Motivation}\label{chapter:motivation}
In present times advanced driver assistance systems, also called ADAS, are common features, even for middle classed driving vehicles. Fully autonomous cars are on the rise and probably only a matter of years before entering series production and running on public streets for the broad mass. These systems usually cover a certain kind of traffic sign recognition. Right now older or not fully equipped cars often lack these kinds of extras, which induces a certain absence of relative safeness over newer cars. As it is not really economic to retrofit these features professionally, the question arises, whether you could add some of them to your vehicle without too much of circumstances, for example with a low-cost device, most drivers already have, like an Android smartphone mounted to the mobile phone bracket. In particular, because these devices have all the hardware, you need to implement this kind of features on, a camera, processors and a display to output the results. So it would actually make sense to have an application, that assists the driver in following the speed limit when steering a vehicle. \newline
Being open source is one of the major benefits of Android over other operating systems for handheld devices, meaning the code is publicly visible and primarily it is free of charges. There are also other gratuitous components beneficial to implement such functionalities, like the OpenCV-library for computer vision and other open-source frameworks for machine learning, like Tensorflow or Caffe(2). So the question comes to mind, how well this low-cost kind of solution actually scores.\newline
The GPU producer Nvidia is currently having a partnership with Audi, developing special system on chip (SOC) solutions, like the Drive PX\footnote{\url{https://www.nvidia.com/en-us/self-driving-cars/drive-px/} (visited on 02.14.2018)}. These high-end products, that are actually built into cars run an ARM-architecture,  belonging to Nvidia's Tegra series\footnote{\url{http://www.nvidia.com/object/tegra.html} (visited on 02.14.2018)}. Another representative of mentioned series is the Nvidia Tegra X1\footnote{\url{http://www.nvidia.com/object/tegra-x1-processor.html} (visited on 02.14.2018)}, that is actually built into the Google Pixel C Android-tablet\footnote{\url{https://www.android.com/tablets/pixel-c/} (visited on 02.14.2018)}. 
As there are apparently similar components built into cars and Android devices, a natural question to ask would be if we can build upon these equalities. \newline
Cooperative Intelligent Transport Systems (C-ITS) is a current research field, aiming to connect vehicles, mobile devices, traffic and transportation infrastructure\cite{kia4sm}. The KIA4SM project wants to achieve cooperative behaviour between independently designed and running systems participating in traffic. Therefore it is not inevitable to change each participating system to make communication possible, rather it is desired to generate a single platform for different devices, which could be solved by Plug-and-Play properties. In this case, device independent actions and dynamic reconfiguration at runtime are going to be the use case, for demonstrating this concept. In order to integrate these diverse components, Electronic Control Units are being used to achieve a flexible integration architecture. 
The paper \cite{kia4sm} presents a new approach to implementing and testing of the described system.
The project's results are, it is possible to lower down the boundaries of each participating component, by replacing the integration architecture by a cooperative and scalable system concept. In the course of this project, a testbed has been presented, including a hybrid simulator, consisting of both physical and virtual components. An imaginable step would be to extend existing testbed with other constituents or features like an Android device, running some sort of driving assistance application. These additional features could comprise for example lane-, vehicle- or traffic sign detection. 
